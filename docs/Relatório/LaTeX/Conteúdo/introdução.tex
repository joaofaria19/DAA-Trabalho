\chapter{Introdução}
\paragraph{}
Este relatório inclui os resultados do trabalho prático realizado no âmbito da unidade curricular de Dados e Aprendizagem Inteligente como forma de aprofundar o conhecimento sobre a matéria lecionada ao longo do semestre. O objetivo deste trabalho foi desenvolver modelos de aprendizagem automática para a resolução de problemas reais.

Os primeiros modelos criados foram desenvolvidos para prever a produção de energia solar. Para isso, foi utilizado um conjunto de \textit{dataset} fornecidos pela equipa docente que contêm informações sobre as condições meteorológicas, a temperatura e a radiação solar dos últimos três anos. Os modelos foram desenvolvidos utilizando técnicas de classificação, tendo como resultado final a previsão da quantidade de energia solar produzida em cada dia.

Os restantes modelos foram desenvolvido para prever o nível de \textit{burnout} em trabalhadores. Para isso, foi utilizado um \textit{dataset} escolhido pelo grupo que contém informações sobre as cargas de trabalho, os escalões e o nível de fadiga mental dos trabalhadores. Os modelos foram desenvolvidos utilizando técnicas de regressão, tendo como resultado final uma pontuação que indica o nível de \textit{burnout} de cada trabalhador.

\chapter{Conclusão}
\paragraph{}
Após a conclusão deste trabalho prático, o grupo considera que os objetivos estabelecidos foram alcançados com sucesso. Os modelos desenvolvidos para os dois \textit{datasets}, um de classificação e outro de regressão, demonstraram um desempenho satisfatório, tanto no que diz respeito à precisão dos resultados como à relevância para os objetivos do projeto.

No que diz respeito ao \textit{dataset} da competição, o modelo de classificação desenvolvido foi capaz de prever com precisão a capacidade de produção de energia na rede elétrica em Braga, em kWh, para cada dia do ano. Este resultado permitiu ao grupo alcançar uma posição respeitável no ranking da competição Kaggle.

No caso do \textit{dataset} do grupo, desenvolvemos modelos de regressão com base nos dados cuidadosamente tratados e analisados, resultando num modelo capaz de prever o nível de burnout entre os trabalhadores. Estamos satisfeitos com a métrica de erro obtida, que consideramos bastante aceitável.

O grupo reconhece que há espaço para melhorias futuras, especialmente no que diz respeito ao refinamento e aprofundamento dos modelos apresentados neste relatório. No entanto, considera que os resultados alcançados são um bom ponto de partida para futuros projetos de Machine Learning.

Além disso, o grupo considera que o trabalho prático proporcionou uma oportunidade valiosa para aprofundar o conhecimento em relação às várias etapas de exploração, tratamento de dados e criação de modelos de aprendizagem automática. O grupo também desenvolveu competências no uso de Python Notebooks e em ambientes de desenvolvimento pelo Anaconda, que permitiram a realização deste projeto.
